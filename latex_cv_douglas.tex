 %%%%%%%%%%%%%%%%%%%%%%%%%%%%%%%%%%%%%%%%%
% Developer CV
% LaTeX Template
% Version 1.0 (28/1/19)
%
% This template originates from:
% http://www.LaTeXTemplates.com
%
% Authors:
% Jan Vorisek (jan@vorisek.me)
% Based on a template by Jan Küster (info@jankuester.com)
% Modified for LaTeX Templates by Vel (vel@LaTeXTemplates.com)
%
% License:
% The MIT License (see included LICENSE file)
%
%%%%%%%%%%%%%%%%%%%%%%%%%%%%%%%%%%%%%%%%%

%----------------------------------------------------------------------------------------
%	PACKAGES AND OTHER DOCUMENT CONFIGURATIONS
%----------------------------------------------------------------------------------------
\documentclass[9pt]{developercv} % Default font size, values from 8-12pt are recommended
\usepackage[utf8]{inputenc}
%----------------------------------------------------------------------------------------

\begin{document}

%----------------------------------------------------------------------------------------
%	TITLE AND CONTACT INFORMATION
%----------------------------------------------------------------------------------------

\begin{minipage}[t]{0.45\textwidth} % 45% of the page width for name
	\vspace{-\baselineskip} % Required for vertically aligning minipages
	% If your name is very short, use just one of the lines below
	% If your name is very long, reduce the font size or make the minipage wider and reduce the others proportionately
	\colorbox{black}{{\HUGE\textcolor{white}{\textbf{\MakeUppercase{Douglas}}}}} % First name
	
	\colorbox{black}{{\HUGE\textcolor{white}{\textbf{\MakeUppercase{Mesquita}}}}} % Last name
	
	\vspace{6pt}
	
	{\huge Software Developer} % Career or current job title
\end{minipage}
\begin{minipage}[t]{0.275\textwidth} % 27.5% of the page width for the first row of icons
	\vspace{-\baselineskip} % Required for vertically aligning minipages
	
	% The first parameter is the FontAwesome icon name, the second is the box size and the third is the text
	% Other icons can be found by referring to fontawesome.pdf (supplied with the template) and using the word after \fa in the command for the icon you want
	\icon{MapMarker}{12}{Dublin, Ireland}\\
	\icon{Phone}{12}{+353 83 8755983}\\
	\icon{At}{12}{\href{mailto:douglasalipio@gmail.com}{douglasalipio@gmail.com}}\\	
\end{minipage}
\begin{minipage}[t]{0.275\textwidth} % 27.5% of the page width for the second row of icons
	\vspace{-\baselineskip} % Required for vertically aligning minipages
	
	% The first parameter is the FontAwesome icon name, the second is the box size and the third is the text
	% Other icons can be found by referring to fontawesome.pdf (supplied with the template) and using the word after \fa in the command for the icon you want
	\icon{Globe}{12}{\href{https://douglasalipio.wordpress.com}{douglas.wordpress}}\\
	\icon{Github}{12}{\href{https://github.com/douglasalipio}{github.com/douglasalipio}}\\
	\icon{Linkedin}{12}{\href{https://www.linkedin.com/in/douglasalipio/}{@douglasalipio}}\\
\end{minipage}

\vspace{0.5cm}

%----------------------------------------------------------------------------------------
%	INTRODUCTION, SKILLS AND TECHNOLOGIES
%----------------------------------------------------------------------------------------

\cvsect{Who Am I?}

\begin{minipage}[t]{0.4\textwidth} % 40% of the page width for the introduction text
	\vspace{-\baselineskip} % Required for vertically aligning minipages
A focused and proactive Android Developer with international experience in Top of Mind mobile brands. Currently searching for personal development and enhancement of professional skills.
\end{minipage}
\hfill % Whitespace between
\begin{minipage}[t]{0.5\textwidth} % 50% of the page for the skills bar chart
	\vspace{-\baselineskip} % Required for vertically aligning minipages
	\begin{barchart}{5.5}
		\baritem{Java}{90}
		\baritem{Kotlin}{90}
		\baritem{Git}{70}
		\baritem{Postgree/MySql}{60}
		\baritem{NoSql}{60}
		\baritem{C Sharp}{50}
		\baritem{Python}{50}
	\end{barchart}
\end{minipage}

%----------------------------------------------------------------------------------------
%	EXPERIENCE
%----------------------------------------------------------------------------------------

\cvsect{Experience}
\newline
\begin{entrylist}
	\entry
		{06/2019}
		{Android Engineer }
		{Oneview. Dublin, Ireland}
{\newline
Tasks include fixing bugs, reviewing code, develop features and participating in meetings to decide the best approach to design the backend service. The application works with Fragments and Activities. Some of the views were created with Android Custom Views and Anko library. Designed database implementing Couchbase light. For dependency injection, it used ToothPick library. The application supports integration tests, Unit test and UI tests. The libraries used for that were Mockito, Roboletric, and JUnit4.
\newline
\newline
 {\texttt{Android SDK}\slashsep\texttt{RESTful API}\slashsep\texttt{Azure DevOps}\slashsep\texttt{Kotlin}\slashsep\texttt{Anko UI}\slashsep\texttt{Crashlytics}\slashsep\texttt{SonarQube}\slashsep\texttt{Git}\slashsep\texttt{Kanban}\slashsep\texttt{MPAndroidChart}\slashsep\texttt{MVP Architecture}}}

\entry
		{2017-2019}
		{Mobile Engineer}
		{Security First. Dublin, Ireland}
{\newline
I worked on developing a new version of Umbrella App, an Irish open source project hosted on GitHub, by following the guidelines of Material Design, creating new features, implementing a MVP pattern and developing automated tests.The new version of the project was written with kotlin 1.3+. In addition, some relevant libraries were used, such as DbFlow, Coroutines, Dagger2, single architecture with Conductor, Espresso, Mockito, etc.
\newline
\newline
 {\texttt{Android SDK}\slashsep\texttt{RESTful API}\slashsep\texttt{Travis CI}\slashsep\texttt{Kotlin}\slashsep\texttt{Android API 16+}\slashsep\texttt{Crashlytics}\slashsep\texttt{Java 8}\slashsep\texttt{Git}\slashsep\texttt{Clean Architecture}\slashsep\texttt{MVVM}}}

\entry
		{Contractor}
		{Android Developer}
		{Spectrum Health. Dublin, Ireland}
{\newline
I worked fixing bugs and settled new features using specific libraries, such as Retrofit, Picasso, Dagger2, and MPAndroidChart. Furthermore, this application consumed data from a server written in PHP language.
\newline
\newline
{\texttt{Android SDK}\slashsep\texttt{RESTful API}\slashsep\texttt{Bug fix}\slashsep\texttt{MVVM Pattern}}}

\entry
		{2014 -- 2017}
		{Senior Software Engineer}
		{Samsung Eletronics. Manaus, Brazil}
{
\newline
\textbf{Accecibility Mode}  Integrated the pioneering team that conducts the company's main studies for the Samsung Gear VR Headset. The team developed an API called Accessibility module for people with disabilities and used Samsung's own GVRf library and Android API to improve the TalkBack, Speech, Zoom and Inverted Colours features.\newline
{\texttt{C++}\slashsep\texttt{Gvrf}\slashsep\texttt{Android SDK}\slashsep\texttt{Git}}
\newline 
\newline  
\textbf{RetailMotion} I worked on the app to introduce Samsung mobile phones to costumers in physical stores. The project had several Android APIs such as Gson, Retrofit, Dagger1, Mockito, Picasso, etc. In order to enhance the quality of the project, Sonarqube and Unit tests were also applied.
  {\texttt{Animation}\slashsep\texttt{Jenkins}\slashsep\texttt{Android SDK}\slashsep\texttt{PForce} \texttt{MVVM Pattern}}
\newline
\newline
 \textbf{GVRf Samples} I developed several sample applications to facilitate the use of the Gvrf open source 3D library. These applications showed how to perform animation, for example, the collision of objects, the identification of objects in the scene, the creation of multiple 3D scenes, etc.
{\texttt{Java}\slashsep\texttt{Gvrf}\slashsep\texttt{Android SDK}\slashsep\texttt{Git}}

}
	\entry
		{2013 -- 2014}
		{Mobile Engineer}
		{Foca Consulting. São Paulo, Brazil}
{\newline
Developed liveTaxi app, were utilised Google Maps API to plot information and add elements to map layers. Additionally, an application server writing Node.js was developed to implement the communication between passenger and driver.\newline
		{\texttt{Android}\slashsep\texttt{Google Maps}\slashsep\texttt{Node.js}\slashsep\texttt{Git}\slashsep\texttt{MongoDB}\slashsep\texttt{AsyncTask}\slashsep\texttt{MVP}}}
\entry
		{2012 -- 2013}
		{Software Engineer}
		{Cin - University of Pernambuco. Recife, Brazil}
{\newline
\textbf{Estadão widget} I worked on the team that developed the application for the biggest Brazilian newspaper Estadão (O Estado de São Paulo). The team used communication with webService and populated the data through an Android View (RemoteView).
\newline
\newline
\textbf{Reader QRCode} I worked on the team that developed a QRCode reader for the Samsung Slate tablet. This project was developed using C Sharp for Windows Phone devices

}
\entry
		{2007 -- 2011}
		{Software Engineer}
		{Grupo Provider. Recife, Brazil}
{\newline
  \textbf{Monitoria} It's a call center attendant monitoring project. The team used Jasper Report and Birt Report to create charts reports. The project used an MVC architecture with Struts, Spring and Hibernate. The app was developed using J2EE.\newline
{\texttt{Java 6}\slashsep\texttt{Spring}\slashsep\texttt{Tomcat}\slashsep\texttt{JPA}\slashsep\texttt{Java JSF}\slashsep\texttt{Java script}\slashsep\texttt{Html}\slashsep\texttt{CSS}} \slashsep\texttt{Hudson}\slashsep\texttt{Sonar}
\newline
\newline
\textbf{Hiper Card}  The application is intended to help employees to input data on tablets such as images and forms. The application has some forms, take pictures and send data to a FTP server.
\newline
{\texttt{Java 6}\slashsep\texttt{Android 1.5}\slashsep\texttt{FIleTransfer}\slashsep\texttt{AsyncTask}\slashsep\texttt{Service}\slashsep\texttt{Camera}\slashsep\texttt{Notification}}

}
\end{entrylist}

%----------------------------------------------------------------------------------------
%	EDUCATION
%----------------------------------------------------------------------------------------

\cvsect{Education}

\begin{entrylist}
	\entry
		{2018 -- 2019}
		{BA Hons in Computer Science}
		{CCT College Dublin}
		{}
	\entry
		{2017 - 2017}
		{General English}
		{Dublin City University}
		{}
	\entry
		{2009 -- 2012}
		{Bachelor's Degree in Information System}
		{Mauricio de Nassau University}
		{}
\end{entrylist}

%----------------------------------------------------------------------------------------
%	ADDITIONAL INFORMATION
%----------------------------------------------------------------------------------------

\begin{minipage}[t]{0.3\textwidth}
	\vspace{-\baselineskip} % Required for vertically aligning minipages

	\cvsect{Languages}
	
	\textbf{Portuguese} - native\\
	\textbf{English} - advanced\\
	\textbf{Spanish} - beginner
\end{minipage}
\begin{minipage}[t]{.7\textwidth}
	\vspace{-\baselineskip} % Required for vertically aligning minipages

	\cvsect{Accomplishments}
	
	\begin{description}
	\item[$\bullet$] \href{http://www.dotazoneapp.com/en/}{http://www.dotazoneapp.com/en/}
\end{description}

\end{minipage}
%\hfill
%\begin{minipage}[t]{0.3\textwidth}
%	\vspace{-\baselineskip} % Required for vertically aligning minipages
	
%	\cvsect{Key assets}
	
%	I love... \lorem
%\end{minipage}
%\hfill
%\begin{minipage}[t]{0.3\textwidth}
%	\vspace{-\baselineskip} % Required for vertically aligning minipages
	
%	\cvsect{Non profit}
	
%	I help... \lorem
%\end{minipage}

%----------------------------------------------------------------------------------------

\end{document}

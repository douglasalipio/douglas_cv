%%%%%%%%%%%%%%%%%%%%%%%%%%%%%%%%%%%%%%%%%
% Developer CV
% LaTeX Template
% Version 1.0 (28/1/19)
%
% This template originates from:
% http://www.LaTeXTemplates.com
%
% Authors:
% Jan Vorisek (jan@vorisek.me)
% Based on a template by Jan Küster (info@jankuester.com)
% Modified for LaTeX Templates by Vel (vel@LaTeXTemplates.com)
%
% License:
% The MIT License (see included LICENSE file)
%
%%%%%%%%%%%%%%%%%%%%%%%%%%%%%%%%%%%%%%%%%

%----------------------------------------------------------------------------------------
%	PACKAGES AND OTHER DOCUMENT CONFIGURATIONS
%----------------------------------------------------------------------------------------
\documentclass[9pt]{developercv} % Default font size, values from 8-12pt are recommended

%----------------------------------------------------------------------------------------

\begin{document}

%----------------------------------------------------------------------------------------
%	TITLE AND CONTACT INFORMATION
%----------------------------------------------------------------------------------------

\begin{minipage}[t]{0.45\textwidth} % 45% of the page width for name
	\vspace{-\baselineskip} % Required for vertically aligning minipages
	
	% If your name is very short, use just one of the lines below
	% If your name is very long, reduce the font size or make the minipage wider and reduce the others proportionately
	\colorbox{black}{{\HUGE\textcolor{white}{\textbf{\MakeUppercase{Douglas}}}}} % First name
	
	\colorbox{black}{{\HUGE\textcolor{white}{\textbf{\MakeUppercase{Mesquita}}}}} % Last name
	
	\vspace{6pt}
	
	{\huge Software Developer} % Career or current job title
\end{minipage}
\begin{minipage}[t]{0.275\textwidth} % 27.5% of the page width for the first row of icons
	\vspace{-\baselineskip} % Required for vertically aligning minipages
	
	% The first parameter is the FontAwesome icon name, the second is the box size and the third is the text
	% Other icons can be found by referring to fontawesome.pdf (supplied with the template) and using the word after \fa in the command for the icon you want
	\icon{MapMarker}{12}{Dublin, Ireland}\\
	\icon{Phone}{12}{+353 83 8755983}\\
	\icon{At}{12}{\href{mailto:douglasalipio@gmail.com}{douglasalipio@gmail.com}}\\	
\end{minipage}
\begin{minipage}[t]{0.275\textwidth} % 27.5% of the page width for the second row of icons
	\vspace{-\baselineskip} % Required for vertically aligning minipages
	
	% The first parameter is the FontAwesome icon name, the second is the box size and the third is the text
	% Other icons can be found by referring to fontawesome.pdf (supplied with the template) and using the word after \fa in the command for the icon you want
	\icon{Globe}{12}{\href{https://douglasalipio.wordpress.com}{douglas.wordpress}}\\
	\icon{Github}{12}{\href{https://github.com/douglasalipio}{github.com/douglasalipio}}\\
	\icon{Linkedin}{12}{\href{https://www.linkedin.com/in/douglasalipio/}{@douglasalipio}}\\
\end{minipage}

\vspace{0.5cm}

%----------------------------------------------------------------------------------------
%	INTRODUCTION, SKILLS AND TECHNOLOGIES
%----------------------------------------------------------------------------------------

\cvsect{Who Am I?}

\begin{minipage}[t]{0.4\textwidth} % 40% of the page width for the introduction text
	\vspace{-\baselineskip} % Required for vertically aligning minipages
I have been a software engineer for over 8+ years specialised in  Mobile Development and Web Development. I am  a open-source enthusiast
and I love build things from scratch. 

\end{minipage}
\hfill % Whitespace between
\begin{minipage}[t]{0.5\textwidth} % 50% of the page for the skills bar chart
	\vspace{-\baselineskip} % Required for vertically aligning minipages
	\begin{barchart}{5.5}
		\baritem{Java}{90}
		\baritem{Kotlin}{80}
		\baritem{C Charp}{50}
		\baritem{Node JS}{60}
		\baritem{Git}{70}
		\baritem{Postgree/MySql}{60}
	\end{barchart}
\end{minipage}

%----------------------------------------------------------------------------------------
%	EXPERIENCE
%----------------------------------------------------------------------------------------

\cvsect{Experience}

\begin{entrylist}
	\entry
		{2017 -- Present}
		{Android Developer}
		{Security First. Dublin, Ireland}
		{I have been working as an Android Developer in an open source project developing a new version following the guidelines of Material Design, create new features, implementing MVP pattern and developing automatic tests.
Umbrella – Umbrella is an Android mobile app developed by Security First that provides human rights defenders with the information on what to do in any given security situation and the tools to do it. There are some libraries that we use such as SQLCipher, Picasso, Crashlytics, AndroidPining, Espresso, Mockito etc.
}
	\entry
		{2017 -- Contractor}
		{Android Developer}
		{Spectrum Health. Dublin, Ireland}
		{I worked as an Android developer to finalize a project which was already being developed by the company team.
Health Coach – we developed an application to help people live a healthy lifestyle and to practice more exercise. We used somes API to build such as Retrofit, Picasso, Dagger and MPAndroidChart. That application consumed data from a server written in PHP language.}
	\entry
		{2014 -- 2017}
		{Software Developer}
		{Samsung Eletronics. Manaus, Brazil}
		{I worked on Research and Development institute for Android applications and participated giving opinions that involved design and software architecture. Below I describes some projects that I worked in.
Accessibility module - I worked with the team that developed a API called Accessibility module for the Samsung Gear VR Glasses. We use samgung’s own GVRf library and Android API to develop the Talk Back, Speech, Zoom and Inverted Colours features.
RetailMode – I worked on the project to introduce the Samsung mobile phones in the stores. We made use of several Android APIs such as Fragments, Animations, Gson, Retrofit, Dagger, Mockito, Picasso etc. We follow the quality of the project using Sonar and developing unit tests.
GVRf Samples – Develop several sample applications to facilitate the use of the Gvrf Api library. These applications showed how to perform animation, the collision of objects, identifica- tion of objects in the scene, creation of multiple 3D scenes, etc.
Smart Controls - We developed a prototype for controlling Samsung’s air conditioners. In this project, we work in a Wi-Fi communication between the device and air-conditioning where it is possible to increase and decrease temperature, turn on and off and scheduling of functionalities.}
	\entry
		{2013 -- 2014}
		{Mobile Developer}
		{Foca Consulting. São Paulo, Brazil}
		{I developed an Android Mobile Application (Live Taxi) and contributed to software design and innovative features. In this project, I utilized Google Maps API to plot information and add elements to map layers.

Project Description:
Application for taxi call, where you can call a taxi and the taxi driver accept the race. To make the communication between driver and passenger was developed an application server writing Node.js}
\entry
		{2012 -- 2013}
		{Software Developer}
		{Cin - University of Pernambuco. Recife, Brazil}
		{I developed applications for Android smartphones and tablets within the Informatics Center of the Federal University of Pernambuco.

Estadão Widget - I worked on the team where we developed the application for the biggest journal Estão We used communication with webService and populated the data through an Android View (RemoteView).

Reader QRCode - I worked on the team where we developed a QRCode reader for the Samsung Slate tablet. This project was developed using C Sharp for Windows Phone devices.}
\entry
		{2007 -- 2011}
		{Software Developer}
		{Grupo Provider. Recife, Brazil}
		{I developed web applications using various technologies and methods provided by the platform and the Java community. The project used a MVC architecture with Struts, Spring and Hibernate. Also, I worked developing the application for Android such as smartphone and tablets.

Project Descriptions:

MONITORIA - I was part of the development team that developed an application to generate reports. We used JasperReport and Birt Report to make this. The application was developed using J2EE

HIPER CARD - I was part of the development team that developed an application for Android. The application was developed to help employees to input data on tablets. The application has some form, take pictures and send data to FTP server.}
\end{entrylist}

%----------------------------------------------------------------------------------------
%	EDUCATION
%----------------------------------------------------------------------------------------

\cvsect{Education}

\begin{entrylist}
	\entry
		{2009 -- 2012}
		{Bachelor's Degree in  Analyse and Information System}
		{University Mauricio de Nassau }
		
	\entry
		{2017}
		{Postgraduate Diploma}
		{A University Name}
		
	\entry
		{2018 -- 2019}
		{Postgraduate Diploma in Computer Science}
		{CCT College}
		
\end{entrylist}

%----------------------------------------------------------------------------------------
%	ADDITIONAL INFORMATION
%----------------------------------------------------------------------------------------

\begin{minipage}[t]{0.3\textwidth}
	\vspace{-\baselineskip} % Required for vertically aligning minipages

	\cvsect{Languages}
	
	\textbf{English} - native\\
	\textbf{German} - proficient\\
	\textbf{Polish} - rudimentary
\end{minipage}
\hfill
\begin{minipage}[t]{0.3\textwidth}
	\vspace{-\baselineskip} % Required for vertically aligning minipages
	
	\cvsect{Hobbies}
	
	I love... \lorem
\end{minipage}
\hfill
\begin{minipage}[t]{0.3\textwidth}
	\vspace{-\baselineskip} % Required for vertically aligning minipages
	
	\cvsect{Non profit}
	
	I help... \lorem
\end{minipage}

%----------------------------------------------------------------------------------------

\end{document}

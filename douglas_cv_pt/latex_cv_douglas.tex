%%%%%%%%%%%%%%%%%%%%%%%%%%%%%%%%%%%%%%%%%
% Developer CV
% LaTeX Template
% Version 1.0 (28/1/19)
%
% This template originates from:
% http://www.LaTeXTemplates.com
%
% Authors:
% Jan Vorisek (jan@vorisek.me)
% Based on a template by Jan Küster (info@jankuester.com)
% Modified for LaTeX Templates by Vel (vel@LaTeXTemplates.com)
%
% License:
% The MIT License (see included LICENSE file)
%
%%%%%%%%%%%%%%%%%%%%%%%%%%%%%%%%%%%%%%%%%

%----------------------------------------------------------------------------------------
%	PACKAGES AND OTHER DOCUMENT CONFIGURATIONS
%----------------------------------------------------------------------------------------
\documentclass[9pt]{developercv} % Default font size, values from 8-12pt are recommended

%----------------------------------------------------------------------------------------

\begin{document}

%----------------------------------------------------------------------------------------
%	TITLE AND CONTACT INFORMATION
%----------------------------------------------------------------------------------------

\begin{minipage}[t]{0.45\textwidth} % 45% of the page width for name
	\vspace{-\baselineskip} % Required for vertically aligning minipages
	
	% If your name is very short, use just one of the lines below
	% If your name is very long, reduce the font size or make the minipage wider and reduce the others proportionately
	\colorbox{black}{{\HUGE\textcolor{white}{\textbf{\MakeUppercase{Douglas}}}}} % First name
	
	\colorbox{black}{{\HUGE\textcolor{white}{\textbf{\MakeUppercase{Mesquita}}}}} % Last name
	
	\vspace{6pt}
	
	{\huge Senior Software Engineer} % Career or current job title
\end{minipage}
\begin{minipage}[t]{0.275\textwidth} % 27.5% of the page width for the first row of icons
	\vspace{-\baselineskip} % Required for vertically aligning minipages
	
	% The first parameter is the FontAwesome icon name, the second is the box size and the third is the text
	% Other icons can be found by referring to fontawesome.pdf (supplied with the template) and using the word after \fa in the command for the icon you want
	\icon{MapMarker}{12}{Recife, Brazil}\\
	\icon{Phone}{12}{+55 81 8755983}\\
	\icon{At}{12}{\href{mailto:douglasalipio@gmail.com}{douglasalipio@gmail.com}}\\	
\end{minipage}
\begin{minipage}[t]{0.275\textwidth} % 27.5% of the page width for the second row of icons
	\vspace{-\baselineskip} % Required for vertically aligning minipages
	
	% The first parameter is the FontAwesome icon name, the second is the box size and the third is the text
	% Other icons can be found by referring to fontawesome.pdf (supplied with the template) and using the word after \fa in the command for the icon you want
	\icon{Globe}{12}{\href{https://douglasalipio.wordpress.com}{douglas.wordpress}}\\
	\icon{Github}{12}{\href{https://github.com/douglasalipio}{github.com/douglasalipio}}\\
	\icon{Linkedin}{12}{\href{https://www.linkedin.com/in/douglasalipio/}{@douglasalipio}}\\
\end{minipage}

\vspace{0.5cm}

%----------------------------------------------------------------------------------------
%	INTRODUCTION, SKILLS AND TECHNOLOGIES
%----------------------------------------------------------------------------------------

\cvsect{Who Am I?}

\begin{minipage}[t]{0.4\textwidth} % 40% of the page width for the introduction text
	\vspace{-\baselineskip} % Required for vertically aligning minipages
Sou engenheiro de software há mais de 10+ anos, especializado em desenvolvimento de sistema. Eu sou um entusiasta de código aberto
e adoro construir coisas do zero.

\end{minipage}
\hfill % Whitespace between
\begin{minipage}[t]{0.5\textwidth} % 50% of the page for the skills bar chart
	\vspace{-\baselineskip} % Required for vertically aligning minipages
	\begin{barchart}{5.5}
		\baritem{Java}{100}
		\baritem{Kotlin}{100}
		\baritem{Git}{80}
		\baritem{Python}{60}
		\baritem{C Charp}{50}
		\baritem{Javascript}{45}
	\end{barchart}
\end{minipage}

%----------------------------------------------------------------------------------------
%	EXPERIENCE
%----------------------------------------------------------------------------------------

\cvsect{Experience}
%------------------------------------------------------------------------------------
%------------------------------------------------------------------------------------
 %----------------------Transact EXPERIENCE------------------------------
 %------------------------------------------------------------------------------------
 %------------------------------------------------------------------------------------
\begin{entrylist}
	\entry
		{2020\\\footnotesize{Atual}}
		{Senior Software Engineer}
		{Transact Campus. Limerick, Ireland}
{

Tenho desenvolvido soluções de comércio do campus para potencializar as compras do campus em soluções integradas de ponto de venda para contas de alunos vinculadas a cartões de identificação do campus e credenciais móveis. Entregamos um produto com todos os recursos em um curto espaço de tempo, permitindo que a empresa conseguisse grandes clientes. As tecnologias incluem credenciais com cartões Biométricos, NFC e Cartão Magnético.
\newline
\newline
Tenho envolvido no gerenciamento de indivíduos tanto em uma base local quanto remota. Contato com o product owner para cumprir prazos apertados e tarefas priorizadas com base na eficiência da equipe.
 	{
 		\texttt{Android SDK}\slashsep
		 \texttt{RESTful API}\slashsep
		 \texttt{Azure}\slashsep
		 \texttt{Kotlin}\slashsep
		 \texttt{Morpho Server}\slashsep
		 \texttt{Jenkins}\slashsep
		 \texttt{Java 8}\slashsep
		 \texttt{Git}\slashsep
		 \texttt{Dagger2}\slashsep
		 \texttt{Scrum}\slashsep
		 \texttt{RxJava}\slashsep
		 \texttt{Jacoco}\slashsep
		 \texttt{Sonarqube}\slashsep
		 \texttt{Junit5}\slashsep
		 \texttt{Sqlite}\slashsep
		 \texttt{Espresso}\slashsep
	}
}

\end{entrylist}

%------------------------------------------------------------------------------------
%------------------------------------------------------------------------------------
 %----------------------Oneview EXPERIENCE------------------------------
 %------------------------------------------------------------------------------------
 %------------------------------------------------------------------------------------
\begin{entrylist}
	\entry
		{2019 -  2020}
		{Senior Android Engineer}
		{Oneview Healthcare. Dublin, Ireland}
{

Trabalhei no projeto Senior Living que é uma plataforma que atua como um ecossistema voltado para o para área da saúde, tendo como objetivo principal a automação de residência para idosos e soluções hospitalares na Austrália e Estados Unidos. As tarefas incluem corrigir bugs, revisar código, desenvolver recursos e melhoria da pipeline do projeto.
\newline
 	{
 		\texttt{Android SDK}\slashsep
		 \texttt{RESTful API}\slashsep
		 \texttt{Travis CI}\slashsep
		 \texttt{Kotlin}\slashsep
		 \texttt{Scrum}\slashsep
		 \texttt{Android API 16+}\slashsep
		 \texttt{Crashlytics}\slashsep
		 \texttt{Java 8}\slashsep
		 \texttt{Git}\slashsep
		 \texttt{Dagger2}\slashsep
		 \texttt{Jetpack components}\slashsep
		 \texttt{RxJava}\slashsep
	}
}
\end{entrylist}
%------------------------------------------------------------------------------------
%------------------------------------------------------------------------------------
 %----------------------Umbrella EXPERIENCE------------------------------
 %------------------------------------------------------------------------------------
 %------------------------------------------------------------------------------------
\begin{entrylist}
	\entry
		{2017 -  2019}
		{Senior Android Engineer}
		{Security First. Dublin, Ireland}
{
Trabalhei no desenvolvimento de uma nova versão do aplicativo Umbrella, projeto pioneiro na comunidade Irlandesa voltado para ajudar a comunidade jornalista ao redor do mundo em manter sua seus dados seguros através da internet. Umbrella é um projeto de código aberto hospedado no GitHub. Minhas principais responsabilidade no projeto foi criar e desenvolver a nova arquitetura do projeto e acompanha o desenvolvimento do projeto com a gerência.

 	{
 		\texttt{Android SDK}\slashsep
		 \texttt{RESTful API}\slashsep
		 \texttt{Travis CI}\slashsep
		 \texttt{Kotlin}\slashsep
		 \texttt{Android API 16+}\slashsep
		 \texttt{Scrum}\slashsep
		 \texttt{Crashlytics}\slashsep
		 \texttt{Java 8}\slashsep
		 \texttt{Git}\slashsep
		 \texttt{Dagger2}\slashsep
		 \texttt{RxJava}\slashsep
	}
}
\end{entrylist}
%------------------------------------------------------------------------------------
%------------------------------------------------------------------------------------
 %----------------------Samsung EXPERIENCE-----------------------------
 %------------------------------------------------------------------------------------
 %------------------------------------------------------------------------------------
\begin{entrylist}
	\entry
		{2014 -  2017}
		{Senior Software Engineer}
		{Samsung Eletronics. Manaus, Brazil}
{
Trabalhei no Departamento de Pesquisa e Desenvolvimento da Samsung Brasil, que tem sede em Manaus no Instituto de Ciência e Tecnologia (SIDIA). Nessa função, contribuí para muitas soluções mobile e web e dei opiniões que envolviam design e arquitetura de software. Além disso, exigia a capacidade de gerenciar vários projetos e trabalhar sob pressão.
\newline
\newline
\textbf{Modo de Acessibilidade} - Integrei a equipe pioneira que realiza as principais pesquisas para o Samsung Gear VR Headset. A equipe desenvolveu o primeiro módulo de Acessibilidade para que pessoas com deficiência visual para que tenham uma melhor experiência com o uso da realidade virtual. Recursos desenvolvidos para Talk-back, Fala, Zoom e Cores Invertidas.

 	{
 		\texttt{Android SDK}\slashsep
 		\texttt{Tizen}\slashsep
		 \texttt{RESTful API}\slashsep
		 \texttt{Sonarqube}\slashsep
		 \texttt{Smart TV}\slashsep
		 \texttt{Unity 3D-VR}\slashsep
		 \texttt{Kotlin}\slashsep
		 \texttt{Android API 16+}\slashsep
		 \texttt{Clean architecture}\slashsep
		 \texttt{Java 7}\slashsep
		 \texttt{Git}\slashsep
		 \texttt{Scrum}\slashsep
		 \texttt{PForce}\slashsep
		 \texttt{Gvrf lib}\slashsep
	}
}
\end{entrylist}
%------------------------------------------------------------------------------------
%------------------------------------------------------------------------------------
 %----------------------Focaconsulting EXPERIENCE----------------------
 %------------------------------------------------------------------------------------
 %------------------------------------------------------------------------------------
\begin{entrylist}
	\entry
		{2013 -  2014}
		{Android Developer}
		{Foca Consulting. São Paulo, Brazil}
{

O Livetaxi foi uma das primeiras aplicações de táxi do mercado brasileiro em 2010. Fui responsável por desenvolver recursos como rastrear frota em tempo real, otimizar operações e oferecer aos passageiros uma experiência superior de reserva de viagens. Principais responsabilidades foram desenvolver novas funcionalidades e correção de erros.
\newline
 	{
 		\texttt{Android SDK}\slashsep
		 \texttt{RESTful API}\slashsep
		 \texttt{Sqlite}\slashsep
		 \texttt{Google Maps}\slashsep
		 \texttt{Java 8}\slashsep
		 \texttt{Git}\slashsep
		 \texttt{Mongo DB}\slashsep
		 \texttt{Scrum}\slashsep
		 \texttt{Node JS}\slashsep
		 \texttt{Async Task}\slashsep
		 \texttt{MVC}\slashsep
	}
}
\end{entrylist}
%------------------------------------------------------------------------------------
%------------------------------------------------------------------------------------
 %----------------------Cin EXPERIENCE------------------------------------
 %-----------------------------------------------------------------------------------
 %-----------------------------------------------------------------------------------
\begin{entrylist}
	\entry
		{2012 -  2013}
		{Android Developer}
		{Cin. Recife, Brazil}
{

Ingressei na empresa para fazer parte da equipe Android da Universidade de Pernambuco no projeto Samsung. A equipe era responsável por entregar soluções de alto valor para o mercado da Samsung.
\newline
\textbf{Estadão widget} - Fiz parte da equipe desenvolvimento para um dos maiores portais de notícias do Brasil,'O Estado de São Paulo'. Um widget que se comunica via Rest API e envia os dados por meio de uma Visualização Android
\newline
\textbf{Reader QRCode} - Trabalhei no QRCode Reader App para tablets da Samsun. Este projeto foi desenvolvido usando C Sharp para dispositivos Windows Phone. Principais funções exigiam desenvolvimento sobre as tecnologias da Microsoft e organizar e priorizar tarefas.

 	{
 		\texttt{Android SDK}\slashsep
		 \texttt{RESTful API}\slashsep
		 \texttt{Sqlite}\slashsep
		 \texttt{Remove View}\slashsep
		 \texttt{C Sharp}\slashsep
		 \texttt{Scrum}\slashsep
		 \texttt{Windows Phone}\slashsep
		 \texttt{Java 5}\slashsep
		 \texttt{Git}\slashsep
		 \texttt{Async Task}\slashsep
		 \texttt{MVC}\slashsep
	}
}
\end{entrylist}
%------------------------------------------------------------------------------------
%------------------------------------------------------------------------------------
 %----------------------Provider EXPERIENCE------------------------------
 %------------------------------------------------------------------------------------
 %------------------------------------------------------------------------------------
\begin{entrylist}
	\entry
		{2007 -  2011}
		{Full-stack}
		{Provider Sistemas. Recife, Brazil}
{

Comecei na empresa como estagiário e me tornei desenvolvedor full-stack em um projeto B2B que é capaz de ajudar o cliente a lidar com sua equipe em um call center. Como desenvolvedor júnior, fui o primeiro funcionário a desenvolver soluções móveis usando o Android 1.5.
\newline
\newline
\textbf{Hipercard App} -O aplicativo teve o intuito de automatizar a emissão de cartões de credito para novos clientes.

 	{
 		\texttt{Android SDK}\slashsep
		 \texttt{HTML}\slashsep
		 \texttt{Java JSF}\slashsep
		 \texttt{Tomcat}\slashsep
		 \texttt{C Sharp}\slashsep
		 \texttt{Spring MVC}\slashsep
		 \texttt{Java 5}\slashsep
		 \texttt{Git}\slashsep
		 \texttt{Async Task}\slashsep
		 \texttt{MVC}\slashsep
		 \texttt{HTML}\slashsep
		 \texttt{CSS}\slashsep
		 \texttt{Hudson CI}\slashsep
		 \texttt{FIleTransfer}\slashsep
		 \texttt{JUnit}\slashsep
		 \texttt{Sonar}\slashsep
		 \texttt{Camera}\slashsep
		 \texttt{Android Notification}\slashsep
	}
}
\end{entrylist}
%----------------------------------------------------------------------------------------
%	EDUCATION
%----------------------------------------------------------------------------------------

\cvsect{Educação}

\begin{entrylist}	
	\entry
		{2018 - 2019}
		{Duration- 1 year}
		{Pós-graduação em Ciências da Computação}
		{Universidade CCT Dublin}
	\entry
		{2017 - 2017}
		{Duration- 1 year}
		{General English}
		{Universidade Dublin City}
	\entry
		{2009 - 2012}
		{Duration- 4 year}
		{Bachelor's Degree em Análise e Desenvolvimento de Sistemas}
		{Universidade Mauricio de Nassau}
\end{entrylist}
%----------------------------------------------------------------------------------------
%	LANGUAGES
%----------------------------------------------------------------------------------------
\begin{minipage}[t]{0.3\textwidth}

	\vspace{-\baselineskip} % Required for vertically aligning minipages

	\cvsect{Languages}
	
	\textbf{Portuguese} - Native\\
	\textbf{English} - Proficient\\
	\textbf{Spanish} - Elementary
\end{minipage}
\hfill
%----------------------------------------------------------------------------------------
%	HOBBIES
%----------------------------------------------------------------------------------------
\begin{minipage}[t]{0.3\textwidth}

	\vspace{-\baselineskip} % Required for vertically aligning minipages
	
	\cvsect{Hobbies}
	
Gosto de ser fisicamente ativo e passo muito tempo jogando Capoeira e aprendendo sobre a história do Brasil através da Capoeira.
\end{minipage}
\hfill
%----------------------------------------------------------------------------------------
%	NON PROFIT
%----------------------------------------------------------------------------------------
\begin{minipage}[t]{0.3\textwidth}
	\vspace{-\baselineskip} % Required for vertically aligning minipages
	
	\cvsect{Non profit}

Quando tenho algum tempo livre, gosto de compartilhar meu conhecimento de informática com as pessoas, começar projetos do zero e ser mais ativo na comunidade de código aberto.
\end{minipage}

\cvsect{Projetos pessoais}
\newline
%----------------------------------------------------------------------------------------
%	DOTAZONE APP
%----------------------------------------------------------------------------------------
\begin{minipage}[t]{0.3\textwidth}

	\vspace{-\baselineskip} % Required for vertically aligning minipages

	\cvsect{Dotazone App}
	
https://play.google.com/store/apps/details?id=br.com.dotazone
\end{minipage}
%----------------------------------------------------------------------------------------
%	METODO 0 APP
%----------------------------------------------------------------------------------------
\newline
\newline
\begin{minipage}[t]{0.3\textwidth}

	\vspace{-\baselineskip} % Required for vertically aligning minipages

	\cvsect{Método 0 App}
	
https://play.google.com/store/apps/details?id=com.br.metodo0.android
\end{minipage}

\end{document}
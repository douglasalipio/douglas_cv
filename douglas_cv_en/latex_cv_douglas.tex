%%%%%%%%%%%%%%%%%%%%%%%%%%%%%%%%%%%%%%%%%
% Developer CV
% LaTeX Template
% Version 1.0 (28/1/19)
%
% This template originates from:
% http://www.LaTeXTemplates.com
%
% Authors:
% Jan Vorisek (jan@vorisek.me)
% Based on a template by Jan Küster (info@jankuester.com)
% Modified for LaTeX Templates by Vel (vel@LaTeXTemplates.com)
%
% License:
% The MIT License (see included LICENSE file)
%
%%%%%%%%%%%%%%%%%%%%%%%%%%%%%%%%%%%%%%%%%

%----------------------------------------------------------------------------------------
%	PACKAGES AND OTHER DOCUMENT CONFIGURATIONS
%----------------------------------------------------------------------------------------
\documentclass[9pt]{developercv} % Default font size, values from 8-12pt are recommended

%----------------------------------------------------------------------------------------

\begin{document}

%----------------------------------------------------------------------------------------
%	TITLE AND CONTACT INFORMATION
%----------------------------------------------------------------------------------------

\begin{minipage}[t]{0.45\textwidth} % 45% of the page width for name
	\vspace{-\baselineskip} % Required for vertically aligning minipages
	
	% If your name is very short, use just one of the lines below
	% If your name is very long, reduce the font size or make the minipage wider and reduce the others proportionately
	\colorbox{black}{{\HUGE\textcolor{white}{\textbf{\MakeUppercase{Douglas}}}}} % First name
	
	\colorbox{black}{{\HUGE\textcolor{white}{\textbf{\MakeUppercase{Mesquita}}}}} % Last name
	
	\vspace{6pt}
	
	{\huge Senior Software Engineer} % Career or current job title
\end{minipage}
\begin{minipage}[t]{0.275\textwidth} % 27.5% of the page width for the first row of icons
	\vspace{-\baselineskip} % Required for vertically aligning minipages
	
	% The first parameter is the FontAwesome icon name, the second is the box size and the third is the text
	% Other icons can be found by referring to fontawesome.pdf (supplied with the template) and using the word after \fa in the command for the icon you want
	\icon{MapMarker}{12}{   , Brazil}\\
	\icon{Phone}{12}{+353 83 8755983}\\
	\icon{At}{12}{\href{mailto:douglasalipio@gmail.com}{douglasalipio@gmail.com}}\\	
\end{minipage}
\begin{minipage}[t]{0.275\textwidth} % 27.5% of the page width for the second row of icons
	\vspace{-\baselineskip} % Required for vertically aligning minipages
	
	% The first parameter is the FontAwesome icon name, the second is the box size and the third is the text
	% Other icons can be found by referring to fontawesome.pdf (supplied with the template) and using the word after \fa in the command for the icon you want
	\icon{Globe}{12}{\href{https://douglasalipio.wordpress.com}{douglas.wordpress}}\\
	\icon{Github}{12}{\href{https://github.com/douglasalipio}{github.com/douglasalipio}}\\
	\icon{Linkedin}{12}{\href{https://www.linkedin.com/in/douglasalipio/}{@douglasalipio}}\\
\end{minipage}

\vspace{0.5cm}

%----------------------------------------------------------------------------------------
%	INTRODUCTION, SKILLS AND TECHNOLOGIES
%----------------------------------------------------------------------------------------

\cvsect{Who Am I?}

\begin{minipage}[t]{0.4\textwidth} % 40% of the page width for the introduction text
	\vspace{-\baselineskip} % Required for vertically aligning minipages
I have been a software engineer for over 10+ years specialised in  Mobile Development and Web Development. I am  an open-source enthusiast
and I love build things from scratch. 

\end{minipage}
\hfill % Whitespace between
\begin{minipage}[t]{0.5\textwidth} % 50% of the page for the skills bar chart
	\vspace{-\baselineskip} % Required for vertically aligning minipages
	\begin{barchart}{5.5}
		\baritem{Java}{90}
		\baritem{Kotlin}{80}
		\baritem{C Charp}{50}
		\baritem{Javascript}{60}
		\baritem{Git}{80}
		\baritem{Postgree/MySql}{60}
	\end{barchart}
\end{minipage}

%----------------------------------------------------------------------------------------
%	EXPERIENCE
%----------------------------------------------------------------------------------------

\cvsect{Experience}
%------------------------------------------------------------------------------------
%------------------------------------------------------------------------------------
 %----------------------Transact EXPERIENCE------------------------------
 %------------------------------------------------------------------------------------
 %------------------------------------------------------------------------------------
\begin{entrylist}
	\entry
		{2020\\\footnotesize{present}}
		{Senior Software Engineer}
		{Transact Campus. Limerick, Ireland}
{

I have been developing campus commerce solutions power campus purchases across integrated point-of-sale solutions for student accounts tied to campus ID cards and mobile credentials. Delivered a fully featured product within a short timeframe enabling the company to land major clients. Technologies include credentials with Biometric, NFC and Magstripe Cards. 
\newline
\newline
Involved managing individuals both on a local and remote basis. Liaised with the product owner to meet tight deadlines and prioritised tasks based on team efficiency.
 	{
 		\texttt{Android SDK}\slashsep
		 \texttt{RESTful API}\slashsep
		 \texttt{Azure}\slashsep
		 \texttt{Kotlin}\slashsep
		 \texttt{Morpho Server}\slashsep
		 \texttt{Jenkins}\slashsep
		 \texttt{Java 8}\slashsep
		 \texttt{Git}\slashsep
		 \texttt{Dagger2}\slashsep
		 \texttt{Scrum}\slashsep
		 \texttt{RxJava}\slashsep
		 \texttt{Jacoco}\slashsep
		 \texttt{Sonarqube}\slashsep
		 \texttt{Junit5}\slashsep
		 \texttt{Sqlite}\slashsep
		 \texttt{Espresso}\slashsep
	}
}

\end{entrylist}

%------------------------------------------------------------------------------------
%------------------------------------------------------------------------------------
 %----------------------Oneview EXPERIENCE------------------------------
 %------------------------------------------------------------------------------------
 %------------------------------------------------------------------------------------
\begin{entrylist}
	\entry
		{2019 -  2020}
		{Senior Android Engineer}
		{Oneview Healthcare. Dublin, Ireland}
{

Worked on Senior Living Application. It is a platform serves as an open ecosystem that provides opportunities to build and deliver applications and standards-based integrations to create new value. Tasks comprise fixing bugs, reviewing code, developing features, and improving the pipeline of the project.
\newline
 	{
 		\texttt{Android SDK}\slashsep
		 \texttt{RESTful API}\slashsep
		 \texttt{Travis CI}\slashsep
		 \texttt{Kotlin}\slashsep
		 \texttt{Scrum}\slashsep
		 \texttt{Android API 16+}\slashsep
		 \texttt{Crashlytics}\slashsep
		 \texttt{Java 8}\slashsep
		 \texttt{Git}\slashsep
		 \texttt{Dagger2}\slashsep
		 \texttt{Jetpack components}\slashsep
		 \texttt{RxJava}\slashsep
	}
}
\end{entrylist}
%------------------------------------------------------------------------------------
%------------------------------------------------------------------------------------
 %----------------------Umbrella EXPERIENCE------------------------------
 %------------------------------------------------------------------------------------
 %------------------------------------------------------------------------------------
\begin{entrylist}
	\entry
		{2017 -  2019}
		{Senior Android Engineer}
		{Security First. Dublin, Ireland}
{
Worked on developing a new version of the Umbrella app, an Irish open source project hosted on GitHub, by following the guidelines of Material Design, creating new features, implementing MVP pattern, and developing automated tests. The new version of the project was written with Kotlin.

 	{
 		\texttt{Android SDK}\slashsep
		 \texttt{RESTful API}\slashsep
		 \texttt{Travis CI}\slashsep
		 \texttt{Kotlin}\slashsep
		 \texttt{Android API 16+}\slashsep
		 \texttt{Scrum}\slashsep
		 \texttt{Crashlytics}\slashsep
		 \texttt{Java 8}\slashsep
		 \texttt{Git}\slashsep
		 \texttt{Dagger2}\slashsep
		 \texttt{RxJava}\slashsep
	}
}
\end{entrylist}
%------------------------------------------------------------------------------------
%------------------------------------------------------------------------------------
 %----------------------Samsung EXPERIENCE-----------------------------
 %------------------------------------------------------------------------------------
 %------------------------------------------------------------------------------------
\begin{entrylist}
	\entry
		{2014 -  2017}
		{Senior Software Engineer}
		{Samsung Eletronics. Manaus, Brazil}
{
Worked at Samsung Brazil’s Research and Development Department, which is head- quartered at the Institute of Science and Technology (SIDIA). In this role, I have contributed to many mobile and web solutions and gave opinions that involved design and software architecture. Also, It required the ability to manage multiple projects and work under pressure.
\newline
\newline
\textbf{Accessibility Mode } - Integrated the pioneering team that conducts the main research for the Samsung Gear VR Headset. The team developed the first  Accessibility module for people with disabilities to have a better experience by using virtual reality. Features developed Talk-back, Speech, Zoom and Inverted Colours.
 	{
 		\texttt{Android SDK}\slashsep
 		\texttt{Tizen}\slashsep
		 \texttt{RESTful API}\slashsep
		 \texttt{Sonarqube}\slashsep
		 \texttt{Smart TV}\slashsep
		 \texttt{Unity 3D-VR}\slashsep
		 \texttt{Kotlin}\slashsep
		 \texttt{Android API 16+}\slashsep
		 \texttt{Clean architecture}\slashsep
		 \texttt{Java 7}\slashsep
		 \texttt{Git}\slashsep
		 \texttt{Scrum}\slashsep
		 \texttt{PForce}\slashsep
		 \texttt{Gvrf lib}\slashsep
	}
}
\end{entrylist}
%------------------------------------------------------------------------------------
%------------------------------------------------------------------------------------
 %----------------------Focaconsulting EXPERIENCE----------------------
 %------------------------------------------------------------------------------------
 %------------------------------------------------------------------------------------
\begin{entrylist}
	\entry
		{2013 -  2014}
		{Android Developer}
		{Foca Consulting. São Paulo, Brazil}
{

Livetaxi was one of the first taxi applications in the Brazilian market. I was responsible to develop features such as track fleet in real-time, optimize operations and offer superior ride booking experience to the passengers.
\newline
 	{
 		\texttt{Android SDK}\slashsep
		 \texttt{RESTful API}\slashsep
		 \texttt{Sqlite}\slashsep
		 \texttt{Google Maps}\slashsep
		 \texttt{Java 8}\slashsep
		 \texttt{Git}\slashsep
		 \texttt{Mongo DB}\slashsep
		 \texttt{Scrum}\slashsep
		 \texttt{Node JS}\slashsep
		 \texttt{Async Task}\slashsep
		 \texttt{MVC}\slashsep
	}
}
\end{entrylist}
%------------------------------------------------------------------------------------
%------------------------------------------------------------------------------------
 %----------------------Cin EXPERIENCE------------------------------------
 %-----------------------------------------------------------------------------------
 %-----------------------------------------------------------------------------------
\begin{entrylist}
	\entry
		{2012 -  2013}
		{Android Developer}
		{Cin. Recife, Brazil}
{

Joined the company to be part of the Android team in the University of Pernambuco. The team was responsible to delivery high Android solutions for the market.
\newline
\newline
\textbf{Estadão widget} - Joined the team to develop a solution for the biggest Brazilian newspaper ’O Estado de São Paulo’. A widget that communicated via Rest API and send the data through an Android View (Remote View). 
\newline
\newline
\textbf{Reader QRCode} - Worked on QRCode Reader App for Samsung Slate tablets. This project was developed using C Sharp for Windows Phone devices. Working at the university required the abilities to organise and prioritise tasks. 

 	{
 		\texttt{Android SDK}\slashsep
		 \texttt{RESTful API}\slashsep
		 \texttt{Sqlite}\slashsep
		 \texttt{Remove View}\slashsep
		 \texttt{C Sharp}\slashsep
		 \texttt{Scrum}\slashsep
		 \texttt{Windows Phone}\slashsep
		 \texttt{Java 5}\slashsep
		 \texttt{Git}\slashsep
		 \texttt{Async Task}\slashsep
		 \texttt{MVC}\slashsep
	}
}
\end{entrylist}
%------------------------------------------------------------------------------------
%------------------------------------------------------------------------------------
 %----------------------Provider EXPERIENCE------------------------------
 %------------------------------------------------------------------------------------
 %------------------------------------------------------------------------------------
\begin{entrylist}
	\entry
		{2007 -  2011}
		{Full-stack}
		{Cin. Recife, Brazil}
{

Started in the company as a Trainee and growing in the company to became a full-stack on a project B2B that is able to help customer to handle their team in a call center. As a junior developer, I was the first employ to develop mobile solutions using Android 1.5.

 	{
 		\texttt{Android SDK}\slashsep
		 \texttt{HTML}\slashsep
		 \texttt{Java JSF}\slashsep
		 \texttt{Tomcat}\slashsep
		 \texttt{C Sharp}\slashsep
		 \texttt{Spring MVC}\slashsep
		 \texttt{Java 5}\slashsep
		 \texttt{Git}\slashsep
		 \texttt{Async Task}\slashsep
		 \texttt{MVC}\slashsep
		 \texttt{HTML}\slashsep
		 \texttt{CSS}\slashsep
		 \texttt{Hudson CI}\slashsep
		 \texttt{FIleTransfer}\slashsep
		 \texttt{JUnit}\slashsep
		 \texttt{Sonar}\slashsep
		 \texttt{Camera}\slashsep
		 \texttt{Android Notification}\slashsep
	}
}
\end{entrylist}
%----------------------------------------------------------------------------------------
%	EDUCATION
%----------------------------------------------------------------------------------------

\cvsect{Education}

\begin{entrylist}	
	\entry
		{2018 - 2019}
		{Duration- 1 year}
		{Postgraduate Diploma in Computer Science}
		{CCT College}
	\entry
		{2017 - 2017}
		{Duration- 1 year}
		{General English}
		{Dublin City University}
	\entry
		{2009 - 2012}
		{Duration- 4 year}
		{Bachelor's Degree in  Analyse and Information System}
		{Mauricio de Nassau University}
\end{entrylist}
%----------------------------------------------------------------------------------------
%	LANGUAGES
%----------------------------------------------------------------------------------------
\begin{minipage}[t]{0.3\textwidth}

	\vspace{-\baselineskip} % Required for vertically aligning minipages

	\cvsect{Languages}
	
	\textbf{Portuguese} - Native\\
	\textbf{English} - Proficient\\
	\textbf{Spanish} - Elementary
\end{minipage}
\hfill
%----------------------------------------------------------------------------------------
%	HOBBIES
%----------------------------------------------------------------------------------------
\begin{minipage}[t]{0.3\textwidth}

	\vspace{-\baselineskip} % Required for vertically aligning minipages
	
	\cvsect{Hobbies}
	
	I enjoy being physically active, and spend a lot of time playing Capoeira and learning about the Brazilian history through Capoeira.
\end{minipage}
\hfill
%----------------------------------------------------------------------------------------
%	NON PROFIT
%----------------------------------------------------------------------------------------
\begin{minipage}[t]{0.3\textwidth}
	\vspace{-\baselineskip} % Required for vertically aligning minipages
	
	\cvsect{Non profit}

When I have some spare time I like to share my computer knowledge with people by starting projects from scratch and being more active in the open-source community.
\end{minipage}

\cvsect{Personal project}
\newline
\newline
%----------------------------------------------------------------------------------------
%	DOTAZONE APP
%----------------------------------------------------------------------------------------
\begin{minipage}[t]{0.3\textwidth}

	\vspace{-\baselineskip} % Required for vertically aligning minipages

	\cvsect{Dotazone App}
	
https://play.google.com/store/apps/details?id=br.com.dotazone
\end{minipage}
%----------------------------------------------------------------------------------------
%	METODO 0 APP
%----------------------------------------------------------------------------------------
\newline
\newline
\begin{minipage}[t]{0.3\textwidth}

	\vspace{-\baselineskip} % Required for vertically aligning minipages

	\cvsect{Método 0 App}
	
https://play.google.com/store/apps/details?id=com.br.metodo0.android
\end{minipage}

\end{document}
